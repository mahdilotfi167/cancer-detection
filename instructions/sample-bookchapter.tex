\documentclass{SBCbookchapter}
\usepackage[utf8]{inputenc}
\usepackage[T1]{fontenc}
\usepackage[english]{babel}
\usepackage{graphicx}

\begin{document}

\section*{Machine Learning}
When most people hear \emph{Machine Learning}, they picture a robot: a dependable butler or a deadly Terminator.
But Machine Learning is not just a futuristic fantasy;
it’s already here. In fact, it has been around for decades 
in some specialized applications, such as spam filters. \\
But... what is machine learning ? \\
here we have a slightly general definition from Arthur Samuel :
\begin{quote}
Machine learning is the field of study that gives computers the ability to learn without being explicitly programmed.
\end{quote}
In other words, Machine Learning is the science or art of programming computers so they can learn from data.\\
Your spam filter is a Machine Learning
program that, given examples of spam
emails (e.g., flagged by users) and examples of regular emails,
can learn to flag spam.\\
In that case set of all emails is the \textit{training set}, and each email is called a \textit{training example}.

\section*{Artificial Intelligence Hierarchy}
Here we have a high level and abstract view of the AI hierarchy.\\
In the top of the hierarchy we have the \textit{Artificial Intelligence}
which includes any technique that enables machines to solve a task in a
way like humans do.\\
In a lower level, we have \textit{Machine Learning}
now we are familiar with this term, so we will not go into details.\\
In the next level, we have \textit{Artificial Neural Networks}
in simple terms, any brain-inspired machine learning model. \\
In the bottom of the hierarchy, we have \textit{Deep Learning}
In the following sections, we'll mostly focus on this level.

\section*{Representation Learning}
A machine learning model transforms its input data
into meaningful outputs, a process that is “learned” 
from exposure to known examples of inputs and outputs. 
Therefore,  the  central  problem  in  machine  learning
and  deep  learning  is  to meaningfully transform  data:
in  other  words,  to  learn  useful representations
of  the  input  data  at hand—representations that 
get us closer to the expected output. \\
Let's look at an example. \\
We have some white and black points in a cartesian plane. \\
Let's say we want to develop an algorithm that takes a coordinates
of a point and output whether it is likely to be white or black. \\

\end{document}