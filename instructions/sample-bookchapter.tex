\documentclass{SBCbookchapter}
\usepackage[utf8]{inputenc}
\usepackage[T1]{fontenc}
\usepackage[english]{babel}
\usepackage{graphicx}

\begin{document}

\section*{Machine Learning}
When most people hear \emph{Machine Learning}, they picture a robot: a dependable butler or a deadly Terminator.
But Machine Learning is not just a futuristic fantasy;
it’s already here. In fact, it has been around for decades 
in some specialized applications, such as spam filters. \\
But... what is machine learning ? \\
here we have a slightly general definition from Arthur Samuel :
\begin{quote}
Machine learning is the field of study that gives computers the ability to learn without being explicitly programmed.
\end{quote}
In other words, Machine Learning is the science or art of programming computers so they can learn from data.\\
Your spam filter is a Machine Learning
program that, given examples of spam
emails (e.g., flagged by users) and examples of regular emails,
can learn to flag spam.\\
In that case set of all emails is the \textit{training set}, and each email is called a \textit{training example}.

\section*{Artificial Intelligence Hierarchy}
Here we have a high level and abstract view of the AI hierarchy.\\
In the top of the hierarchy we have the \textit{Artificial Intelligence}
which includes any technique that enables machines to solve a task in a
way like humans do.\\
In a lower level, we have \textit{Machine Learning}
now we are familiar with this term, so we will not go into details.\\
In the next level, we have \textit{Artificial Neural Networks}
in simple terms, any brain-inspired machine learning model. \\
In the bottom of the hierarchy, we have \textit{Deep Learning}
In the following sections, we'll mostly focus on this level.

\section*{Representation Learning}
A machine learning model transforms its input data
into meaningful outputs, a process that is “learned” 
from exposure to known examples of inputs and outputs. 
Therefore,  the  central  problem  in  machine  learning
and  deep  learning  is  to meaningfully transform  data:
in  other  words,  to  learn  useful representations
of  the  input  data  at hand—representations that 
get us closer to the expected output. \\
Let's look at an example. \\
We have some white and black points in a cartesian plane. \\
Let's say we want to develop an algorithm that takes a coordinates
of a point and output whether it is likely to be white or black.

\section*{What Is Cancer ?}
Cancer is a disease in which some of the body’s 
cells grow uncontrollably and spread to other parts of the body. \\ 
Cancer can start almost anywhere in the human body, 
which is made up of trillions of cells. Normally, 
human cells grow and multiply to form new cells as the 
body needs them. When cells grow old or become 
damaged, they die, and new cells take their place. \\
Sometimes this orderly process breaks down, and
abnormal or damaged cells grow and multiply when
they shouldn’t. These cells may form tumors, which
are lumps of tissue. Tumors can be cancerous or
not cancerous. 

\section*{Types of Cancers}
Here we have an overview of major cancer types. \\
This is not a complete introduction to cancers, 
so we will not go into details for all of them. \\
But for this context, we will focus on a special type of blood cancer
called \textit{Acute Lymphoblastic Leukemia} (ALL). \\
because in the following sections, we have a practical example
of a deep learning model that can detect ALL from blood cells.

\section*{Acute Lymphoblastic Leukemia}
Acute lymphocytic leukemia (or acute lymphoblastic leukemia, ALL)
is a type of cancer of the blood and bone marrow — 
the spongy tissue inside bones where blood cells are made. \\
The word "acute" in acute lymphocytic leukemia comes
from the fact that the disease progresses rapidly and
creates immature blood cells, rather than mature ones.
The word "lymphocytic" 
refers to the white blood cells (B-cells or T-cells) called lymphocytes, 
which ALL affects. \\
Normal lymphoblasts develop into mature, B-cells or T-cells.\\
In ALL, both the normal development of some 
lymphocytes and the control over the number of 
lymphoid cells become defective.

\section*{Cancer Detection}
But... why cancer detection is important ? \\
Diagnosing cancer at its earliest stages often provides the
best chance for a cure.

\section*{Practical Example}
Here we have a practical example of detecting ALL from blood cell images. \\
But before we start, let's introduce libraries \& tools we've used for this example, if you wanna
work on this field, you should use them. \\
The library we've used for this example is \textit{TensorFlow} which is a free and open-source software library for artificial intelligence. \\
TensorFlow also includes a library for deep learning called \textit{Keras} which is an open-source neural-network library written in Python. \\
The first tool is \textit{Google Colab} which is a free cloud service
that allows you to run Jupyter notebooks on Google's servers. \\
Jupyter notebook is an environment that allows you to run Python code blocks. \\
The second one is \textit{Kaggle} which is a platform for data science competitions. \\
Also it has a lot of datasets and tutorials. \\
This is not a complete machine learning course, ...

\section*{In Real Life}
Over the past several years, researchers have developed AI tools that have the potential to make cancer imaging faster, more accurate, and even more informative. And that’s generated a lot of excitement. \\
for example, AI-based computer programs have been used to help doctors interpret MRIs for more than 20 years, but research in this area is quickly evolving. \\
Although scientists are churning out AI tools for cancer imaging, the field is still nascent and many questions about the practical applications of these tools remain unanswered. \\
A recent study, for example, showed that a machine learning algorithm trained to predict cancer outcomes zeroed in on the hospital where the tumor image was taken, rather than the patient’s tumor biology.

\end{document}